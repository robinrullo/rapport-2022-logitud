\documentclass{rapportUHA40}
\usepackage{lipsum}

\title{Rapport RULLO Robin 2022} %Titre du fichier

\begin{document}

%----------- Informations du rapport ---------

\logoentreprise{logos/logitud-big.png}

\titre{Refonte et amélioration d'une application de cartographie SIG} % Titre du fichier
\pied{Refonte et amélioration d'une \\ application de cartographie SIG} % Pied de page

\typerapport{Rapport de stage} % Type de rapport
\trigrammemention{RULLO Robin} % Pour le bas de la page
\filiere{Licence professionnelle \\ développeur informatique} % Nom de la filière
\promotion{Promotion 2020 – 2021}
\niveau{UHA 4.0.2}

\eleve{Robin RULLO}

\dates{14/02/2022 – 12/08/2022}

% Informations tuteurs écoles
\tuteurecole{
  \textsc{Mounir ELBAZ} \\
  mounir.elbaz@uha.fr
}

\tuteurentreprise{
  \textsc{EL Mahdi SAHI} \\
  m-sahi@logitud.fr
}

%----------- Initialisation -------------------

\fairepagedegarde%Créer la page de garde
\fairemarges%Afficher les marges

\pagenumbering{roman}

%----------- Remerciements -------------------

\vspace*{\stretch{1}}
\begin{center}
  \begin{abstract}
    Je souhaite tout d’abord remercier Guillaume LOOS, responsable des
    développements, qui m'a intégré dans l'équipe de Recherche et Développements.

    Je tiens à remercier El Mahdi SAHI, responable du service R\&D ainsi que mon
    collègue Mohamed TAMA, ingénieur géomaticien et développeur dans le service,
    toujours disponnible et qui m'a pas mal forcé la main pour rédiger ce rapport
    en temps et en heure. Merci à eux de m’avoir suivi et fait confiance tout au
    long du stage.

    Je remercie également tous les développeurs, les hotliners, les formateurs, et
    plus généralement tous ceux qui ont pris le temps de répondre a mes questions
    et avec qui j'ai pu échanger des connaissances pour ainsi progresser.
  \end{abstract}
\end{center}
\vspace*{\stretch{1}}
\newpage

%------------ Table des matières ----------------

\tabledematieres% Créer la table de matières

%------------ Corps du rapport ----------------

%------------ Introduction ----------------

\pagenumbering{arabic}
\section{Introduction}
Logitud Solutions SAS est une entreprise spécialisée dans l'édition de
logiciels pour les collectivités, dans les domaines de la population et de la
sécurité, depuis 30 ans. Afin de rester leader du marché, elle doit faire face
aux logiciels de la concurrence et est forcée de se maintenir à jour
technologiquement. Elle a amorcé depuis quelques années la réécriture de ses
applications lourdes, nécéssitant une installation sur un poste de travail,
vers des applications web fonctionnant sur un navigateur web. Ces applications
traitent un lot données géo-référencées, lesquelles nécessitant quelques fois
des traitements. \\

M'étant spécialisé dans les systèmes d'information géographique depuis le début
de la formation à l'UHA 4.0, je souhaitais, en déposant une candidature
spontannée, assoir mes connaissance en géomatique dans un milieu professionnel,
entouré d'experts pouvant me guider et echanger leurs connaissances. \\\\

Le rapport courvre la présentation de l'organisme dans lequel le stage a eu
lieu, puis une présentation du stage. En fin, une conclusion boucle le rapport.

L'application web SIG permettant de gérer les contextes géographiques de la
suite d'applications n'est plus adapté aux nouveaux besoin et n'est pas
suffisament robuste pour être déployé en production.

%------------- Commandes utiles ----------------

\section{Quelques commandes}

Voici quelques commandes utiles:

%------ Pour insérer et citer une image centralisée -----

\insererfigure{logos/uha-big.png}{3cm}{Légende de la figure}{Label de la figure}
% Le premier argument est le chemin pour la photo
% Le deuxième est la hauteur de la photo
% Le troisième la légende
% Le quatrième le label

Ici, je cite l'image~\ref{fig: Label de la figure}

%------- Pour insérer et citer une équation --------------

\begin{equation} \label{eq: exemple}
  \rho + \Delta = 42
\end{equation}

L'équation~\ref{eq: exemple} est cité ici. \\

%------- Pour insérer des blocks de code --------------
Affichage de blocs de codes:
\begin{minted}[autogobble, linenos, frame = single]{java}
public class HelloWorld {
\end{minted}

Reprendre la dernière numérotation:
\begin{minted}[autogobble, linenos, firstnumber = last, frame = single]{java}
  public static void main(String[] args) {
        // Prints "Hello, World" to the terminal window.
        System.out.println("Hello, World!");
    }


}
\end{minted}

It's also possible to link directly any word or \hyperlink{thesentence}{any
  sentence} in you document.

If you read this text, you will get no information. Really? Is there no
information?

For instance \hypertarget{thesentence}{this sentence}.

\end{document}