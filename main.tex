\documentclass{rapportUHA40}
\usepackage[toc]{glossaries}
\usepackage{makeidx}
\usepackage{pdfpages}
\newglossaryentry{applicationsmetier}
{
  name=applications métier,
  description={Applications destinées à l'utilisateur final pour répondre à son besoin issu de son métier}
}
\newglossaryentry{sig}
{
  name=SIG,
  description={Système d'information géographique}
}
\newglossaryentry{reverse-geocoding}
{
  name=reverse-geocoding,
  description={Attribution d'une adresse à partir de coordonnées géographiques}
}
\newglossaryentry{SCRUM}
{
  name=framework SCRUM,
  description={Lié à la méthodologie Agile. Il améliore la productivité des équipes tout en permettant une optimisation du produit grâce aux retours réguliers des clients}
}
\newglossaryentry{agile}
{
  name=agile,
  description={La méthodologie Agile est un modèle d'organisation de projet qui place le client à son cœur}
}
\newglossaryentry{proxy}
{
  name=proxy,
  description={Hôte intermédiaire se plaçant entre deux hôtes pour faciliter ou surveiller leurs échanges}
}
\newglossaryentry{open-source}
{
  name=open-source,
  description={Le logiciel est libre et peut être utilisé par tous}
}
\newglossaryentry{appshell}
{
  name=App Shell,
  description={Squelette de l'application}
}
\newglossaryentry{wrapper}
{
  name=wrapper,
  description={Le wrapper est une entité qui encapsule et masque la complexité sous-jacente d'une autre entité au moyen d'interfaces bien définies. Dans le cas d'un wrapper HTTP, il encapsule et masque la couche HTTP et renvoie les données après traitement ou brut.}
}
\makeglossaries{}
\makeindex

\title{Rapport RULLO Robin 2022} %Titre du fichier

\begin{document}

%----------- Informations du rapport ---------

\logoentreprise{logos/logitud-big.png}

\titre{Refonte et amélioration d'une application de cartographie SIG} % Titre du fichier
\pied{Refonte et amélioration d'une \\ application de cartographie SIG} % Pied de page

\typerapport{Rapport de stage} % Type de rapport
\trigrammemention{RULLO Robin} % Pour le bas de la page
\filiere{Diplôme universitaire \\ D.U. 4.0.2} % Nom de la filière
\promotion{Promotion 2020 – 2021}
\niveau{UHA 4.0.2}

\eleve{Robin RULLO}

\dates{14/02/2022 – 12/08/2022}

% Informations tuteurs écoles
\tuteurecole{
  \textsc{Mounir ELBAZ} \\
  mounir.elbaz@uha.fr
}

\tuteurentreprise{
  \textsc{EL Mahdi SAHI} \\
  m-sahi@logitud.fr
}

%----------- Initialisation -------------------

\fairepagedegarde%Créer la page de garde
\fairemarges%Afficher les marges

\pagenumbering{roman}

%----------- Remerciements -------------------

\vspace*{\stretch{1}}
\begin{center}
  \begin{abstract}
    Je souhaite tout d’abord remercier Guillaume LOOS, responsable des
    développements, qui m'a permis d'intégrer dans l'équipe de Recherche et Développements.
    \vspace{1cm}

    Je tiens à remercier El Mahdi SAHI, responsable du service R\&D ainsi que mon
    collègue Mohamed TAMA, ingénieur géomaticien et développeur, toujours
    disponible et qui m'a pas mal forcé la main pour rédiger ce rapport en temps et
    en heure. Merci à eux de m’avoir suivi et fait confiance tout au long du stage.
    \vspace{1cm}

    Je remercie également tous les développeurs, les hotliners, les formateurs, et
    plus généralement tous ceux qui ont pris le temps de répondre à mes questions
    et avec qui j'ai pu échanger et ainsi progresser.
  \end{abstract}
\end{center}
\vspace*{\stretch{1}}
\newpage

%------------ Table des matières ----------------

\renewcommand{\baselinestretch}{0.9}\normalsize
\tabledematieres% Créer la table de matières
\renewcommand{\baselinestretch}{1.0}\normalsize

%------------ Corps du rapport ----------------
\setcounter{figure}{0}% Reset du conteur de figures

%------------ Introduction ----------------

\pagenumbering{arabic}
\section{Introduction}
Logitud Solutions est une entreprise spécialisée dans l'édition de logiciels
pour les collectivités, dans les domaines de la population et de la sécurité,
depuis 30 ans. Le web s'étant beaucoup développé ses dernières années, les
logiciels sont devenus difficiles et couteux à maintenir. L'entreprise a alors
amorcé depuis quelques années la réécriture de ses applications lourdes liées
au secteur de la sécurité (polices municipales) qui nécessitent une
installation sur un poste de travail, vers des applications web fonctionnant
sur un navigateur web ainsi que des applications pour mobiles.

L'ensemble des applications clients lourds puis applications web font appel à
des données géoréférencées. Le rôle de l'application web \gls{sig} \textsc{Map
  Manager} est l'administration les données géographiques. L'entreprise souhaite
faire évoluer l'application et la rendre plus ergonomique. Pour répondre aux
nouveaux besoins, une refonte de l'application est nécessaire. \\

Attiré et spécialisé dans les systèmes d'informations géographiques depuis le
début de la formation à l'UHA 4.0, je souhaitais, en déposant ma candidature,
assoir mes connaissances en géomatique dans un milieu professionnel, entouré
d'experts pouvant me guider et échanger leurs connaissances.

\vspace{2cm}

Le rapport couvre la présentation de l'organisme dans lequel le stage a eu
lieu, puis une présentation du stage. En fin, une conclusion boucle le rapport.

\newpage
%------------ Organisme d'accueil ----------------

\section{Organisme d'accueil}
Logitud Solutions est une société par actions simplifiée dont le siège social
est situé dans la ZAC du Parc des Colline de Mulhouse – Didenheim. Elle compte
également deux autres agences: l'agence centre à Saint-Avertin (37), et
l’agence sud à Saint-Rémy-de-Provence (13). Elle est constituée de 90 employés
dont 30 développeurs. \\ Son secteur d'activité est l'édition d'outils
numériques destinés aux collectivités locales (communes, communautés de
communes, villes). Elle distribue aujourd'hui trois gammes de logiciels.
Chronologiquement:

\begin{description}
  \item[La gamme population] – Elle est tournée vers la gestion administrative des
    collectivités. Elle facilite le travail des agents d'état civil et leurs
    échanges avec les administrés avec des logiciels tels que Siècle, SuffrageWeb,
    Éternité, Avenir ou encore Populis.
  \item[La gamme sécurité] Elle est orientée vers la gestion des métiers de la police
    (organisation, gestion des fourrières, géo-verbalisation) et prévention de la
    délinquance (géo-prévention délinquance et des incivilités).
  \item[La gamme e-administration] Elle regroupe des services en ligne et mobiles à
    l'attention des citoyens.
\end{description}

L'entreprise est un acteur majeur du marché dans cette gamme de logiciels. Elle
équipe un tier des villes de plus de 5 000 habitants en produits d’état-civil,
et les quatre cinquièmes en produits de sécurité (polices municipales).\\

La méthodologie de travail adopté par les équipes de développement métier suit
les principes du \gls{SCRUM} couplé à la méthodologie \Gls{agile}. Cependant
l'équipe de R\&D dont est rattaché le pôle de développement cartographique
bénéficie d'énormément de liberté dans son management et aucune stratégie de
management spécifique n'est appliquée.

\newpage
%------------ Le stage ----------------
\section{Le stage}
\subsection{Présentation du contexte}

Afin de s'adapter aux services proposés par la concurrence ainsi qu'aux
technologies actuelles, Logitud est repartie de zéro en réécrivant les
applications qui était alors jusque-là des applications clients lourds en
applications web (clients légers). De nombreuses applications des suites
métiers (gamme population, sécurité, etc\ldots) interagissent avec des données
géoréférencées. \textsc{Map Manager} est l'application \gls{sig} permettant
d’administrer ces différentes données. Elle est maintenue par l’équipe en
charge de l’infrastructure géographique.

\subsection{La géomatique et le \gls{sig}}
La géomatique ou « la géographie appliquée à l'informatique », est la
discipline regroupant les pratiques qui permettent de collecter, analyser et
diffuser des données géographiques par l'informatique. L'application
\textsc{Map Manager} s'inscrit principalement dans la diffusion des données
mais aussi dans la collecte avec la possibilité de création et modification
d'objets géographiques.

Afin de se repérer et de localiser l'information sur la surface terrestre, il
est nécessaire d'utiliser un système de position comprenant:
\begin{description}
  \item[la définition d'un référentiel] dont son but est de fournir aux utilisateurs
    des points stables et matérialisés par des bornes de coordonnées connues. En
    France, la norme est le référentiel RGF (Réseau Géodésique Français) 93.
    Cependant, la norme internationale est le WGS (World Geodetic System) 84,
    utilisé par les Américains et associé au système GPS\@.
    \insererfigure{figures/rgf93rbf.jpeg}{6.5cm}{Les 1032 sites constituant le
      réseau de base français pour maintenir le RGF93}{Les sites constituant le
      réseau de base français}

  \item[le choix d'un système de projections et de coordonnées] dont le but est de
    projeter l'image de la terre assimilé à un ellipsoïde en une surface plane.
    Encore une fois, en France, nous utilisons la projection Lambert 93. Cependant,
    la plupart des cartes numériques mises à dispositions du grand public utilisent
    la projection WGS84 Web Mercator.
\end{description}

\insererfigure{figures/lambertCC_mercator84_merged.png}{3cm}{
  Projection Lambert conique conforme à gauche, Mercator à droite\\
  \copyright{} Tobias Jung
}{Exemple de projections}

\textsc{Map Manager} est une application \gls{sig}. Le \gls{sig}, pour Système d'Information Géographique, est un système d’information
qui intègre, stocke, analyse et affiche l'information géographique qui est de
la donnée localisée sur le territoire. Cette donnée peut être:
\begin{description}
  \item[Géométrique]: La donnée décrit la forme et la position (points, lignes,
  polygones), repéré dans un système de projection retenu et donc superposable
  avec d'autres données.
  \item[Attributaire]: La donnée attributaire fournis des informations complémentaire
  permettant de caractériser la donnée géométrique, de type numérique, texte,
  date, etc\ldots
  \item[Semiologique]: La donnée sémiologique fournis les informations pour représenter
  les données géométriques sur la carte (taille, couleur, pictogrammes,
  etc\ldots).
\end{description}

\subsection{Étude de l'existant}
L'application \textsc{Map Manager} existe déjà mais ne correspond plus aux
besoins en termes d'ergonomie et de fonctionnalités. C'est une Single Page Web
Application (SPA – application web à page unique). A mon arrivée, elle avait
déjà subie trois refontes car plusieurs besoins qui n'avaient pas été exprimés
au début du développement se sont ajoutés au fur et à mesure de l'utilisation
de l'application. \\

L'application a également son homonyme en librairie Angular, appelée
\textsc{Map-Viewer}, dont son but est la consultation des données
cartographiques dans les \gls{applicationsmetier}. Cette librairie, est incluse
dans la librairie de composants graphiques partagés utilisée par l'entreprise,
\textsc{WebUI-Core}, afin d'uniformiser et simplifier l'affichage des données
dans les autres applications. Cette librairie est une librairie Angular avec un
composant qui permet d'avoir la même carte et interactions que celles
développées dans \textsc{Map Manager}. Il a également fallu faire une
réécriture pour prendre en compte les mêmes besoins que \textsc{Map Manager}
mais cette fois en gardant la même base car il ne fallait pas produire de
modification cassantes nécessitant une adaptation du côté des application
l'ayant implémenté.

J’ai réalisé la première semaine un document rendant (cf.
\hyperlink{ANNEX2}{annexe 2}) compte de l’état des fonctionnalités développées
en suivant l’approche du « Manual Testing », pratique qui consiste à tester
toutes les fonctionnalités manuellement sur l’interface web afin de tester
entièrement les fonctionnalités. Ce document m’a permis de définir le point de
départ et de comprendre le besoin des clients. Il m'a permis de mettre en
évidence les points qui suivent.

\subsubsection{Le lien entre les géométries et les objets dans les applications}
Il y a trois catégories d'objets géoréférencés qui sont constitués des trois
différentes géométries présentées précédemment:
\begin{itemize}
  \item Les \textbf{secteurs} représentés par une géométrie polygonale
  \item Les \textbf{POI}s (Point of interest – Points d'intérts) représentés par le
        point
  \item Les \textbf{itinéraires} représentés par la géométrie linéaire. \\
\end{itemize}

Dans le domaine métier géographique, les collègues ont fait le choix
d'organiser ces géométries dans un ensemble de types. Un type est défini par un
nom (ex: « Stationnement payant »), une couleur, ainsi qu'un icône. La couleur
peut être surchargée dans l'objet géométrique que contiendra le type tandis que
l'objet possèdera forcément l'icône du type. Ces types peuvent être rattachés à
un ou plusieurs contextes métiers propres à un module d'une application de la
suite logicielle. Les contextes métiers sont gérés par le service
\textsc{Labels} qui est commun à toutes les applications, qui permet de
personnaliser les données de l'interface utilisateur à la guise du client en
mettant à disposition des listes personnalisées.

\subsubsection{État de l'art}
Toute l'infrastructure SIG de l'entreprise est déjà en place et fonctionnelle.
Elle est composée de plusieurs mini-services divisé en deux thématiques, cf.
figure~\ref{fig: Architecture SIG}. La première partie entourée en orange est
l'administration des objets géographiques. C'est également celle sur laquelle
j'ai travaillé. La deuxième est la recherche d'adresses.

La première partie est composée de quatre modules
\begin{itemize}
  \item \textsc{GeoToolbox}, le serveur de
        traitement et stockage cartographique est le module principal.
  \item \textsc{Geoserver} est un serveur de données cartographiques open-sources.
        Il est consommé par le serveur GeoToolbox pour réaliser tous les traitements
        géospatiaux comme le \gls{reverse-geocoding}.
  \item \textsc{Map Printer} est un moteur de rendu cartographique permettant de générer
        des cartes à partir de modèles de mise en page prédéfinis.
  \item \textsc{Map Manager}.
\end{itemize}

La seconde partie est composée de trois modules: Le serveur \textsc{addresses}
\gls{proxy} les services de recherche d'adresses. Pour la recherche d'adresses
en France, nous utilisons un serveur et moteur de recherche \textsc{Addok} avec
les données de la Base Nationale d'Adresses fournie par l'État et régulièrement
mise à jour. Pour les recherches d'adresses en dehors de la France, nous
consommons un service gratuit basé sur les données open-source d'OpenStreetMap.

\insererfigure{figures/architecture-SIG.png}{11cm}{Architecture
  \gls{sig}}{Architecture SIG}

\subsection{Définition du besoin}
\textsc{Map Manager} répond au besoin d'administration de secteurs,
itinéraires et poi. Dans l'ancienne version, les objets ont été séparé en trois
entités distinctes affichés sur trois vues différentes. Aujourd'hui, nous
souhaitons réunir ces trois objets en une seule entité afin de visualiser les
objets de trois types de géométries sur la même carte.
\insererfigure{figures/homescreen.png}{4cm}{Choix de la vue selon le type de
  géométrie}{Map-Viewer}

Pour afficher les objets, il est nécessaire de sélectionner un ou plusieurs
types ou contextes. Des géométries s'affichent et dans l'ancienne version, un
tableau s'ouvre sur la partie médiane basse de l'écran, listant les géométries
correspondantes aux critères de recherche. Sur de petits écrans, la place est
monopolisée par le menu de recherche et ce tableau:
\insererfigure{figures/no\_space.png}{6cm}{Place monopolisée par les
  menus}{Manque de place sur l'ancienne version de l'application}

En ce qui concerne ensuite la modification des géométries, on peut soit en
ajouter, soit les modifier avec des outils de la carte très basiques. On peut
également consulter les données qui y sont référencées.

Une page paramètre dans l'application permet de réaliser un SCRUD (Search,
Create, Read, Update, Delete) sur les types métiers. Une autre section de
l'application permet également d'importer une collection de données
géographiques dans un type métier déjà existant (à nouveau avec la contrainte
de séparation des types de géométrie). J'ai également pu remonter un certain
nombre de comportements indésirés ou bugs.

Pour interagir avec les objets géographiques, il faut passer par le serveur
GeoToolbox. Il permet de créer les types, les objets géographiques et de les
modifier par la suite. Le serveur GeoToolbox est développé en parallèle et
indépendamment de \textsc{Map Manager} par un collègue expert géomaticien.

Avant de commencer la réécriture de l'application, nous avons déterminé les
principaux besoins sur lesquels se focaliser pour cette nouvelle version:
\begin{enumerate}
  \item L'application doit être dans un premier temps iso-fonctionnelle.
  \item Afficher toutes les catégories d'objets géographique sur une seule carte.
  \item Toutes les fonctionnalités dans une seule vue carto-centrée.
  \item L'application doit être ergonomique pour l'utilisateur qui fréquente peu
        l'application et qui n'est pas à l'aise avec les \gls{sig}.
  \item L'application devra utiliser les librairies communes aux autres applications.
\end{enumerate}

Pour permettre à l'application d'unifier les secteurs, les POI et les
itinéraires en une seule entité, un collègue s'est attelé en parallèle à la
refactorisation des entités géographiques dans le serveur GeoToolbox.

\subsection{Développement du projet et difficultés rencontrées}
Dans cette partie, nous allons voir en détail les différentes étapes de
développements pour la réalisation de la nouvelle version de l'application.

\subsubsection{Prototypage}
J'ai débuté la refonte par la réalisation d'une maquette pour l'interface
utilisateur sur un logiciel de prototypage, \textbf{Adobe XD}. Mon collègue m'a
suggéré de m'inspirer du site \href{https://www.mapillary.com/app/}{Mapillary},
cf. figure~\ref{fig: Mapillary}.
\insererfigure{figures/screen_mapillary.png}{6cm}{\href{https://www.mapillary.com/app/}{Mapillary}}{Mapillary}

Cependant, le design de la maquette ne correspondait pas à celui des autres
applications de Logitud. J'ai donc continué le prototypage en me basant sur le
système de design Clarity UI de VMWare afin d'uniformiser l'interface avec
celles des différentes applications car la librairie de composant partagée de
l'entreprise, \textsc{WebUI-Core}, utilisée par les autres applications.

Nous avons ensuite organisé une réunion avec le référent UI/UX, le responsable
du service R\&D, le responsable des développements et également un responsable
de projet de la hotline Logitud afin de valider l'implémentation du besoin
métier ainsi que l'ergonomie à l'utilisation. Nous avons retenu la proposition
suivante (cf. figure \ref{fig: Proposition de maquette retenue}):
\insererfigure{figures/maquette\_retained\_proposition.png}{6cm}{Proposition
  retenue}{Proposition de maquette retenue}

\subsubsection{Choix des technologies et structure}
Toutes les applications de l'entreprise sont basées sur le Framework Web
Angular 9 et \textsc{Map Manager} n'en fait pas exception. Nous ne pouvons pas
utiliser Angular 13 (dernière version stable à ce jour) car toutes les
applications et librairies communes, notamment \textsc{WebUI-Core}, sont
bloquées à cette version. Monter la version des projets Angular n'est pas
envisagé pour l'instant à cause de certaines librairies externes non supportées
et la quantité de modification à réaliser pour mettre à jour la version
d'Angular.

En ce qui concerne le Web-mapping, nous avons pris la décision de continuer
d'utiliser la librairie OpenLayers permettant d'afficher la carte dynamique.
Nous en avons une bonne connaissance, elle est open-source, très mature ainsi
que suffisante pour répondre à nos besoins actuels.

Nous avons décidé pour la réécriture, d'initialiser un nouveau projet Angular
et de tout réimplémenter en suivant l'architecture que nous avions défini afin
de rendre l'application maintenable:
\begin{minted}[autogobble, frame = single]{text}
src/app/
+-- config
+-- core
|   +-- http
|   +-- layout
+-- enums
+-- interfaces
+-- modules
|   +-- geo-entity
+-- services
|   +-- external
+-- shared
|   +-- directives
|   +-- map
|   +-- modal
+-- utils
\end{minted}

\subsubsection{Implémentation de la maquette}
\textsc{Map Manager} interagis avec de nombreux services de la suite logicielle que nous
découvrirons plus tard. Nous avons décidé de se focaliser de prime abord sur
l'intégration de la maquette dans l'application Angular. C'est pourquoi tous
les modèles de données fournis par les services externes ont été mockés: une
réponse \og{type} \fg{} avec des données statiques est renvoyé au lieu de
consommer réellement le service externe.

\paragraph{Création des composants de la maquette}\mbox{}\\
L'application étant la seule carto-centré dans la suite d'applications, aucun
composant n'est intégré à la librairie \textsc{WebUI-core} excepté
l'\gls{appshell}.
\insererfigure{figures/proto_app-shell.png}{4cm}{\Gls{appshell}}{App Shell}

\newpage
J'ai donc débuté par l'ajout des composant de la barre latérale avec des
données mockés:

\insererfigure{figures/proto_sidebar.png}{6cm}{Barre
  latérale}{Barre latérale}

En premier lieu, j'ai ajouté les \og{cards} \fg{} représentant une entité de la
liste de types ou d'objets géographiques:
\insererfigure{figures/proto_cards.png}{2cm}{Capture d'écran d'une carte d'un
  type et en-dessous celle d'un secteur}{Capture d'écran d'une carte d'un type et
  en-dessous celle d'un secteur}

J'ai poursuivi par l'implémentation du composant d'affichage des objets
géographiques en utilisant les cards créés précédemment. J'ai également ajouté
le composant de recherche d'objets géographiques à partir de types et
contextes: \insererfigure{figures/proto_type-select.png}{1.5cm}{Composant de
  recherche de types à gauche et composant de visualisation du contenu du type à
  droite}{Capture d'écran Composant de recherche de types à gauche et composant
  de visualisation du contenu du type à droite}

Je me suis confronté à un problème survenant lors du changement de vue dans la
sidebar vers le composant d'affichage des objets géographiques. En effet, le
changement provoque la destruction du composant d'affichage des types. Tous les
filtres de recherches étaient donc détruits et réinitialisés lorsqu'on
naviguais en arrière vers les types. Je me suis retrouvé dans la même
problématique qu'était les développeurs du réseau social FaceBook il y a
quelques années et dont découle l'architecture basée sur les flux.

\insererfigure{figures/flux_architecture.png}{4cm}{Architecture basée sur les
  flux}{Architecture basée sur les flux}

Le principe de flux simplifie la dépendance des composants entre eux. Dans le
cas ci-présent, l'état des filtres est gardé dans un store global dans
l'application qui est détruit avec l'application. Le composant et sa vue est
mis à jour lorsque cet état est modifié avec le pattern
\textit{Publish-Subscribe}.

Pour résoudre ma problématique, j'ai proposé de mettre en place cette
architecture dans l'application. Cependant j'étais le seul à l'aise avec une
architecture basé sur les flux dans l'entreprise. Nous avons cependant deux
autres solutions qui sont plus cohérentes et moins couteuses à mettre en place
dans Angular. Les solutions à mon problème sont de garder l'état du filtre dans
le composant parent ou dans un service. J'ai donc retenu l'utilisation du
composant parent pour conserver l'état des filtres car c'était la solution la
plus simple et rapide à implémenter.\\

\paragraph{Implémentation des actions et de la donnée}\mbox{}\\
J'ai ensuite implémenté le composant metadata qui est la colonne vertébrale de
l'application. C'est lui qui a la charge de récupérer et traiter les données
pour ensuite les distribuer aux différents composants. Il fait le lien entre
les recherches qui consomment les services externes, les filtres sur les
résultats et l'affichage sur la carte. C'est également dans ce composant que
l'implémentation du filtre \og{highlight} \fg{} qui permet de mettre en
surbrillance un élément sur la carte est réalisé. Il a été laborieux à mettre
en place car il y a trois états de surbrillance à gérer qui sont liés entre eux
comme suit:

\begin{enumerate}
  \item L'état \textbf{normal} lorsqu'aucun élément n'est en surbrillance.
  \item L'état de \textbf{surbrillance} lorsqu'on sélectionne un élément, tous les
        autres sont désactivés. De plus, la sélection est incrémentale
  \item L'état \textbf{désactivé} lorsqu'un ou plusieurs éléments sont en surbrillance,
        les autres éléments ont l'état désactivé.
\end{enumerate}

\paragraph{Implémentation des services externes}\mbox{}\\
Une fois l'implémentation des vues et des filtres terminés, je suis passé à
l'implémentation des services Angular afin de remplacer les mocks de données
par la consommation de services externes. L'entreprise fait systématiquement un
\gls{wrapper} pour chaque service afin d'abstraire la couche HTTP\@. Le wrapper
fournie des méthodes qui renvoient la donnée correspondante aux filtres. J'ai
débuté par l'implémentation des méthodes du \gls{wrapper} \gls{sig} permettant
de requêter sur le serveur cartographique afin de récupérer la liste de types
et d'effectuer la recherche d'objets contenus dans ces types. \\

\paragraph{Implémentation de la carte}\mbox{}\\
Maintenant que les objets sont récupérés, il faut les afficher sur la carte.

\insererfigure{figures/screen_map-manager.png}{7cm}{Map Manager}{Map Manager}

L'application consomme le serveur GeoToolbox qui lui retourne des collections
d'objets au format GeoJSON, cf. \hyperlink{ANNEX1}{annexe 1}. J'ai utilisé le
pattern \textit{subject/subscriber} implémenté par d'Angular. Le composant
metadata publie les objets récupérés par le wrapper puis le composant de la
carte les affiche si les conditions d'affichage sont réunies.

Pour afficher une source de données sur la carte dans le format GeoJSON utilisé
par \textsc{GeoToolbox}, il suffit de les lire avec le \og{Reader} \fg{} de ce
dernier qui est fourni par la librairie Openlayers. La librairie les ajouter à
une couche vectorielle qui elle-même est ajoutée à l'instance de la carte. Nous
avons décidé d'implémenter le pattern \textit{Adapter} avec un service jouant
le rôle d'adapteur et contenant l'implémentation des méthodes de la librairie
OpenLayers afin de simplifier le changement de librairie cartographique si
cette dernière ne répond plus à nos besoins.

Maintenant que les objets sont ajoutés à la carte, il faut définir leur style.
Les metadonnées des objets géographiques, comme évoqué lors de la présentation
de la géomatique, contiennent des données sémiologiques, définissant en
l'occurrence la couleur et le pictogramme de l'objet. Le serveur encode la
couleur en hexadécimal qui est supporté nativement par la librairie. En
revanche, le pictogramme provient des différentes librairies
\textsc{FontAwesome}, \textsc{ClarityIcons} ou le service \textsc{Labels} avec
des icônes personnalisés par le client. La librairie OpenLayers contraint à
l'utilisation d'icônes encodés en base64:
\textit{data:image/svg+xml;base64,\ldots}. Dans le cas des icônes du service
labels, c'est simple, on le récupère dans le bon format. En revanche, dans le
cas des deux librairies d'icônes, on récupère uniquement son nom. Il a donc
fallu dans un premier temps le récupérer au format SVG à partir de son nom, sur
certains réaliser des traitements, puis le convertir dans le format attendu.
J'ai réalisé de nombreux essais pour récupérer l'icône de la librairie
\textsc{ClarityIcons}, car en effet, c'est assez facile de récupérer le code
SVG de l'icône, mais il faut le traiter par la suite:
\insererfigure{figures/clr_convert.png}{1cm}{Icon \og{Clock} \fg{} de Clarity
  non traité vers l'icône traité}{Traitement de l'icône Clarity}

Le SVG, est un format d'image basé sur le XML\@. Cela signifie que l'on peut le
manipuler afin de supprimer les nœuds XML contenant les géométries normalement
cachées par le style du document (CSS) qui n'est pas pris en compte par les
canvas utilisés par OpenLayers pour afficher la carte ainsi que les icônes.
Pour le reste, j'ai pu me servir du code de la version précédente afin
d'afficher les objets sur la carte.

Aussi important que de pouvoir afficher les objets sur la carte, il faut
pouvoir les créer en les dessinant et plus généralement, exécuter des actions à
partir de la carte. C'est ce sur quoi j'ai ensuite travaillé. J'ai implémenté
les \og{controls OpenLayers} \fg{} permettant de réaliser des d'actions sur la
carte et en dehors tels que le sélecteur de fonds cartographiques, les
différents outils de dessin et de modification, le bouton pour ouvrir la barre
latérale ainsi que les interactions de la carte (zoom, rotation, recentrage sur
le contour de la ville).

L'enregistrement des dessins a été particulièrement ardu. Un des besoins est la
possibilité de dessiner plusieurs géométries dans un seul objet géographique.
Le \textit{GeoJSON} le permet avec un objet poly-géométrique, cependant
OpenLayers produit des objets à géométrie unique. Il a fallu implémenter des
méthodes de conversions de types de géométries de Point$\,\to\,$MultiPoint,
LineString$\,\to\,$MultiLineString, Polygon$\,\to\,$MultiPolygon.

Je me suis intéressé ensuite à une demande d'évolution. L'application doit
permettre la modification de géométries avec différents outils comme la mise à
l'échelle, la rotation, le déplacement de coordonnées ou encore la suppression
d'objet. J'ai fait cela à partir d'interactions déjà existantes dans une
extension de la librairie: \textsc{ol-ext}. Cette évolution doit permettre
également de modifier plusieurs objets à la fois et de types différents.

\paragraph{Import de géométries}\mbox{}\\
La dernière fonctionnalité manquante dans l'application avant de pouvoir la
mettre en production est l'import d'objets cartographiques. Deux manières d'import
ont été demandés:
\begin{itemize}
  \item Importer des objets dans un type existant.
  \item Importer et fusionner les géométries des objets dans un objet existant.
\end{itemize}

J'ai fait le choix de prendre de la dette technique en introduisant de la
complexité inutile afin de livrer une première version de l'application
rapidement car la présentation et mise en production étaient imminentes.

Tout d'abord, j'ai créé un composant pour téléverser le fichier contentant les
objets à importer par glissement (drop) avec des vérifications nécessaires pour
s'assurer que le fichier pourra ensuite être traité (taille, format). J'ai
ensuite créé un composant pour sélectionner la projection de la donnée. La
correspondance des données attributaires et les géométries sont gérés par le
composant d'import qui va sérialiser les données afin de les afficher sur la
carte. OpenLayers met à disposition des développeurs des sérialiseurs pour les
formats \textit{GeoJSON} et \textit{KML}. Pour supporter l'import du format
Shapefile qui est un standard de nombreux logiciels \gls{sig}, il a fallu
passer par une librairie externe permettant de convertir auparavant les données
en \textit{GeoJSON}.

Ensuite, il faut traiter les métadonnées associées. Pour ça, un formulaire
permet de définir la correspondance entre les métadonnées des objets importés
et les métadonnées de notre système:
\insererfigure{figures/screen_mm_import.png}{7cm}{Vue de l'import
  d'objets}{Import d'objets géographiques dans l'application}

En ce qui concerne l'interface utilisateur, la plus grande partie avait déjà
été codé dans l'ancienne version de l'application. Il a été nécessaire de
séparer l'enregistrement en fonction de la destination de l'import: dans de
nouveaux objets à ajouter à un type existant ou bien fusionner un objet
existant. \\

\paragraph{Documentation de l'application}\mbox{}\\
La dernière étape avant de créer la version a été la génération du changelog
(cf. \hyperlink{ANNEX3}{annexe 3}), la documentation des changements du projets
entre les différentes versions. J'ai depuis le début de la réécriture de
l'application, fait le choix de suivre la convention de commit d'Angular connu
pour rendre l'historique de versionnement explicite. Par exemple avec l'ajout d'une
fonctionnalité dans les objets géographique: \textit{feat(geofeatures-component): add layer-cards component}

La convention concorde de plus avec la convention de versionnage des
application \textbf{SemVer} qui est utilisée par l'entreprise, ayant trois
chiffres: le premier pour les modifications cassantes, le deuxième pour les
nouvelles fonctionnalités et le troisième pour les corrections. J'ai mis en
place un script comparant les commits depuis le précédant tag de version afin
de générer le changelog et de monter la version de l'application
automatiquement. En fonction du type des commits (\textit{feat, fix, perf, ci,
  refactor, docs, build}), la partie correspondante de la version est accru.

\subsection{Déploiement}
Pour tester l'application lors du développement, nous avons mis en place un
environnement d'intégration continue afin de déployer régulièrement
l'application sur l'environnement de test.

Suite à un non-versionnement de l'API du serveur cartographique GeoToolbox et
une modification cassante, \textsc{Map Manager} a été déployé en production un
peu plus rapidement que planifié initalement, en Mai. En effet suite à une
demande d'évolution dans le serveur cartographique GeoToolbox pour le nouveau
\textsc{Map Manager} produisant un changement cassant dans l'API du serveur
backend GeoToolbox qui unifie maintenant les objets de types de géométries
différents en une seule entité afin de pouvoir les récupérer à partir d'un seul
end-point. L'ancienne version n'était plus fonctionnelle et à ce moment, toutes
les anciennes fonctionnalités avaient été implémentées dans la nouvelle version
de l'application.

Le processus de déploiement est géré par l'équipe en charge de
l'infrastructure. Pour déployer l'application en production, il faut ouvrir une
demande sur le site du support Logitud.

\subsection{Améliorations et perspectives}
Bien que l'application contienne plus de fonctionnalités que la précédente,
plusieurs évolutions sont encore à implémenter et d'autres envisagées.

Sur la partie cartographie, nous souhaitons plus tard permettre à l'utilisateur
de personnaliser son interface en lui permettant de sauvegarder des préférences
d'affichage. Nous aimerions également permettre au client de configurer son
propre flux cartographique pour les fonds cartographiques.

De nouvelles fonctionnalités sont également planifiées dans la prochaine
version. L'utilisation d'un serveur de moteur de rendu cartographique
permettant de générer la carte affiché dans un fichier à partir d'un modèle
défini est actuellement en cours d'implémentation dans la librairie
textsc{Map-Viewer} et sera également implémentée dans \textsc{Map Manager}.

De plus, de nouvelles évolutions sont imaginées mais pas encore programmées. Il
serait intéressant d'implémenter le visualiseur d'image
\href{https://www.mapillary.com/app/}{Mapillary}. Il permet de visualiser la
rue sur des photos partagées par des contributeurs. Nous voudrions visualiser
et dessiner des objets géographiques sur la photo.

\section{Conclusion}
À mon arrivée dans le service, mon collègue m’a confié de diverses tâches dans
les différents mini-services cartographiques, me permettant ainsi de découvrir
leurs différentes fonctionnalités. L'absence de documentation m'a permis de
m’extravertir et d’échanger avec mes collègues. Elle m’a également permis
d’apprendre l’origine des choix et décisions techniques réalisés, d’enrichir mes
connaissances sur le domaine métier et le besoin des clients. J’ai également eu
l’occasion d’amener l’utilisation de certaines bonnes pratiques permettant de gagner
du temps et d’augmenter la maintenabilité dans d'autres projets.
Le projet \textsc{Map Manager} dont j'ai eu la tâche de refonte et d'ajout de nouvelles
fonctionnalités, exploité par les différentes \gls{applicationsmetier}, m’a également permis
de découvrir l’application de la gamme sécurité destinée à la police municipale,
« MunicipolWEB 2 », sur laquelle portera ma prochaine mission avec l’enchainement
sur un contrat de professionnalisation. L'application \textsc{Map Manager} est aujourd'hui déployée sur
l'environnement de production de l'entreprise.

%------------ FIN ----------------
\newpage
% Glossaire
\printglossaries{}
% Bibliographie
\begin{thebibliography}{9}
  \bibitem{bworld}
  IGN 2008, \emph{Le repère RGF93 et la projection Lambert-93}, \url{http://aiweb.techfak.uni-bielefeld.de/content/bworld-robot-control-software/}
\end{thebibliography}
\newpage

\section*{Résumé}
\addcontentsline{toc}{section}{Résumé}

Logitud est une entreprise éditant des logiciels répondant aux besoins des
métiers des collectivités locales. Une évolution des besoins de l'application
web de \gls{sig} à nécessité la réécriture complète de l'application. Afin de
s'assurer de répondre aux nouveaux besoins ainsi que de s'assurer de ne pas
perdre de fonctionnalités déjà existantes, un travail de documentation de
l'existant et de prototypage a été nécessaire avant le développement. Le
développement s'est terminé par le déploiement de la nouvelle version sur
l'environnement de production pour la totalité les clients de cette suite
logiciel et aucun retour négatif n'est pour l'instant connu.

\section*{Mots clés}
\begin{itemize}
  \item Refonte
  \item Cartographie – \gls{sig}
  \item Framework Web – Angular
\end{itemize}

\clearpage
\appendix
\pagenumbering{gobble}
\section*{ANNEXE}
\addcontentsline{toc}{section}{Annexe}

\subsection*{\hypertarget{ANNEX1}{ANNEXE 1 – Exemple d'une collection d'objets contenant un POI au format GeoJSON}}
\begin{minted}[autogobble, linenos, frame = single]{JSON}
  {
  "features": [
    {
      "geometry": {
        "coordinates": [
            7.3306349332097,
            47.75105398476878
        ],
        "type": "Point"
      },
      "type": "Feature",
      "properties": {
        "color": "#01B7D6",
        "provider": "CUST",
        "name": "camera",
        "description": "",
        "id": "CAMERA_1653034125850_1",
        "type": {
          "icon": {
            "value": "ICONE/DEFAUT/AMPOULEA",
            "source": "LABELS",
          },
          "id": "STATIONNEMENT_1638527874599_8",
          "name": "Stationnement",
          "color": "#064bf3",
          "provider": "CUST",
          "isFavorite": false,
          "contexts": []
        },
        "category": "POI",
      }
    },
  ],
  "type": "FeatureCollection"
}
\end{minted}

\clearpage
\includepdf[pages=1,pagecommand=\subsection*{\hypertarget{ANNEX2}{ANNEXE 2 – Document réalisé rendant compte des fonctionnalités de la version avant refonte}}, scale=0.8, offset=0 -1.5cm]{annex/analyse_fonctionnalites_existantes.pdf}
\includepdf[pages=2-, pagecommand={}, scale=0.8]{annex/analyse_fonctionnalites_existantes.pdf}
\includepdf[pages=1,pagecommand=\subsection*{\hypertarget{ANNEX1}{ANNEXE 3 – Changelog généré automatiquement pour la version 1.2.1}}, scale=0.8, offset=0 -2.5cm]{annex/CHANGELOG.pdf}

\end{document}