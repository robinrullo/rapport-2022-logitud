\documentclass{rapportUHA40}
\usepackage{lipsum}
\title{Rapport RULLO Robin 2022} %Titre du fichier

\begin{document}

%----------- Informations du rapport ---------

\logoentreprise{logos/logitud-big.png}

\titre{Refonte et amélioration d'une application de cartographie SIG} % Titre du fichier
\pied{Refonte et amélioration d'une \\ application de cartographie SIG} % Pied de page


\typerapport{Rapport de stage} % Type de rapport
\trigrammemention{RULLO Robin} % Pour le bas de la page
\filiere{Licence professionnelle \\ développeur informatique} % Nom de la filière
\promotion{Promotion 2020-2021}
\niveau{UHA 4.0.2}

\eleve{Robin RULLO}

\dates{14/02/2022 - 12/08/2022}

% Informations tuteurs écoles
\tuteurecole{
    \textsc{Mounir ELBAZ} \\
    mounir.elbaz@uha.fr
} 

\tuteurentreprise{
    \textsc{EL Mahdi SAHI} \\
    m-sahi@logitud.fr 
}

%----------- Initialisation -------------------

\fairepagedegarde %Créer la page de garde        
\fairemarges %Afficher les marges

%----------- Abstract -------------------
\vspace*{\stretch{1}}
\begin{center}
	\begin{abstract}
        \lipsum[1-2]
    \end{abstract}
\end{center}
\vspace*{\stretch{1}}
\newpage

%------------ Table des matières ----------------

\tabledematieres % Créer la table de matières

%------------ Corps du rapport ----------------


%------------ Introduction ----------------

\section{Introduction} 
\lipsum[3-4] % Effacer cette ligne et écrire le texte souhaité




\newpage
%------------ Related work ----------------

\section{Travaux connexes}
\lipsum[1-2] 





\newpage
%------------ Methodology ----------------

\section{Méthodologie}
\lipsum[2-3]





\newpage
%------------ Application à l'analyse de texte ----------------

\section{Section 1}



%------------- Commandes utiles ----------------

\section{Quelques commandes}

Voici quelques commandes utiles :

%------ Pour insérer et citer une image centralisée -----

\insererfigure{logos/uha-big.png}{3cm}{Légende de la figure}{Label de la figure}
% Le premier argument est le chemin pour la photo
% Le deuxième est la hauteur de la photo
% Le troisième la légende
% Le quatrième le label

Ici, je cite l'image \ref{fig: Label de la figure}


%------- Pour insérer et citer une équation --------------

\begin{equation} \label{eq: exemple}
\rho + \Delta = 42
\end{equation}

L'équation \ref{eq: exemple} est cité ici. \\

%------- Pour insérer des blocks de code --------------
Affichage de blocs de codes:
\begin{minted}[autogobble, linenos, frame = single]{java}
public class HelloWorld {
\end{minted}

Reprendre la dernière numérotation:
\begin{minted}[autogobble, linenos, firstnumber = last, frame = single]{java}
  public static void main(String[] args) {
        // Prints "Hello, World" to the terminal window.
        System.out.println("Hello, World!");
    }

}
\end{minted}


\end{document}
